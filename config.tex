% Se define un color gris desde su código RGB
\definecolor{gris}{RGB}{220,220,220}

\setcounter{secnumdepth}{3} % Para permitir numerar las sub-subsecciones

% Modifica el nombre de los índices al castellano
\addto\captionsspanish{
  \renewcommand{\contentsname}{Índice de contenido}
  \renewcommand{\listfigurename}{Índice de figuras}
  \renewcommand{\listtablename}{Índice de tablas}
}

% Formateo de los nombres de los apartados:
\titleformat{\chapter}[block]
  {\normalfont\Huge\bfseries\singlespacing}{\thechapter.}{1em}{\Huge}
\titlespacing*{\chapter}{0pt}{-62pt}{0pt}

\titleformat{\section}[block]
  {\normalfont\large\bfseries}{\thesection.-}{4pt}{\large}
\titlespacing*{\section}{0pt}{\baselineskip}{0pt}

\titleformat{\subsection}[block]
  {\normalfont\normalsize\bfseries}{\thesubsection.-}{4pt}{\normalsize}
\titlespacing*{\subsection}{0pt}{0pt}{0pt}

\titleformat{\subsubsection}[block]
  {\normalfont\normalsize\bfseries}{\thesubsubsection.-}{4pt}{\normalsize}
\titlespacing*{\subsubsection}{0pt}{0pt}{0pt}

\def\tablename{Tabla}

\fancypagestyle{plain}{%
  \fancyhf{}
  \fancyhead[L]{\includegraphics[height=16mm]{square.png}
    \hspace{1em} \Longstack[l]{
      \textbf{NOMBRE DE LA ASIGNATURA} \newline
      \textbf{Título del Trabajo}}}
  \fancyhead[R]{\bfseries{Hoja \thepage \hspace{1pt} de \pageref{LastPage}}}
  \fancyfoot[C]{Juan Francisco Mier Montoto}
  \renewcommand{\headrulewidth}{0pt}% default is 0pt
  \renewcommand{\footrulewidth}{0.4pt}% default is 0pt
}

\pagestyle{fancy}

\restylefloat{table}
